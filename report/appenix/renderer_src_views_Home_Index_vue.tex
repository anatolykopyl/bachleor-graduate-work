\subsection{Модуль <<renderer/src/views/Home/Index.vue>>}
\begin{lstlisting}[language=vue]
<script setup lang="ts">
import { ref, computed, toRaw } from "vue";
import TraceChart from "./TraceChart.vue";
import GenerateModal from "./GenerateModal.vue";
import ActionsModal from "./ActionsModal.vue";
import type TSor from "src/models/sor";
import { useStore } from "@renderer/store";
import { storeToRefs } from "pinia";

const store = useStore();
const { sor } = storeToRefs(store);

const lossThr = ref<string>();
const reflThr = ref<string>();
const teoThr = ref<string>("0.2");

const showBuiltin = ref<boolean>(true);
const showNaive = ref<boolean>(false);
const showTeo = ref<boolean>(true);
const showTeoVal = ref<boolean>(false);

const sorEdited = ref(false);
const justSavedToLib = ref<boolean>(false);
const showActionsModal = ref();
const clickX = ref<number>();

const openFile = async (): Promise<void> => {
  const dialogConfig = {
    properties: ["openFile"],
  };

  const result = await window.electron.ipcRenderer.invoke("dialog", "showOpenDialog", dialogConfig);
  const filePath = result.filePaths[0];

  sor.value = await window.electron.ipcRenderer.invoke("openSOR", filePath) as TSor;
  lossThr.value = String(sor.value.info.FxdParams.lossThr);
  reflThr.value = String(sor.value.info.FxdParams.reflThr);
  justSavedToLib.value = false;
  sorEdited.value = false;

  console.log(sor.value);
};

const inLibrary = computed((): boolean => {
  if (!sor.value) {
    return false;
  }

  if (justSavedToLib.value) {
    return true;
  }

  const library = window.electron.store.get("library");
  return library.some((s) => s.info.filename === sor.value?.info.filename);
});

const saveToLibrary = (): void => {
  if (!sor.value) {
    return;
  }

  justSavedToLib.value = true;
  const library = window.electron.store.get("library");
  library.push(toRaw(sor.value));
  window.electron.store.set("library", library);
};

const updateLoss = (value: string): void => {
  if (!sor.value) {
    return;
  }

  sor.value.info.FxdParams.lossThr = parseFloat(value);
};

const updateRefl = (value: string): void => {
  if (!sor.value) {
    return;
  }

  sor.value.info.FxdParams.reflThr = parseFloat(value);
};

function crop(x: number): void {
  if (!sor.value) {
    return;
  }

  sor.value.trace = sor.value.trace.filter((dp) => dp.x < x);
  sorEdited.value = true;
}

function generate({
  att, noiseScale,
}: { att: number, noiseScale: number }): void {
  if (!sor.value) {
    return;
  }

  while(sor.value.trace.length < sor.value.info.DataPts["num data points"]) {
    const lastDP = sor.value.trace[sor.value.trace.length - 1];

    const step = sor.value.info.FxdParams.resolution * 0.001;

    const stepAtt = att * step;
    const noise = noiseScale * (Math.random() - .5) * stepAtt;

    // Potentially causing excessive rerenderings
    sor.value.trace.push({
      x: lastDP.x + step,
      y: lastDP.y - stepAtt + noise,
    });
  }
}

function addEvent({
  x,
  type,
  loss,
  reflection,
}: {
  x: number,
  type: "reflection" | "loss",
  loss: number,
  reflection: number
}): void {
  if (!sor.value) {
    return;
  }

  const length = 8;

  const eventStart = sor.value.trace.findIndex((dp) => dp.x > x);
  const eventEnd = eventStart + length;

  for (let i = eventStart; i < eventEnd; i++) {
    if (type === "loss") {
      sor.value.trace[i].y = sor.value.trace[i].y - loss;
    } else {
      sor.value.trace[i].y = sor.value.trace[i].y + reflection;
    }
  }

  for (let i = eventEnd; i <= sor.value.trace.length; i++) {
    sor.value.trace[i].y = sor.value.trace[i].y - loss;
  }
}

function handleClickTrace(x: number): void {
  showActionsModal.value = true;
  clickX.value = x;
}
</script>

<template>
  <v-app-bar title="Главная">
    <template #append>
      <v-btn @click="store.navigate('library')">
        Библиотека файлов
      </v-btn>
    </template>
  </v-app-bar>

  <v-main>
    <TraceChart
      v-if="sor"
      :sor="sor"
      :teo-thr="parseFloat(teoThr)"
      :show-builtin="showBuiltin"
      :show-naive="showNaive"
      :show-teo="showTeo"
      :show-teo-val="showTeoVal"
      :sor-edited="sorEdited"
      @click-trace="handleClickTrace"
    />

    <div
      v-if="sor"
      class="controls"
    >
      <div class="controls__file">
        <div>
          Открыт файл
          <b class="controls__file__filename">{{ sor.info.filename }}</b>
        </div>
        <div>
          {{ sor.info.FxdParams.wavelength }}
        </div>
        <div>
          {{ new Intl.DateTimeFormat('ru-RU').format(new Date(sor.info.FxdParams.date)) }}
        </div>
        <div>
          ID кабеля {{ sor.info.GenParams.cableId }}
        </div>
        <div>
          ID волокна {{ sor.info.GenParams.fiberId }}
        </div>
        <v-btn
          class="controls__file__open"
          @click="openFile"
        >
          Открыть файл
        </v-btn>
        <v-btn
          v-if="!inLibrary"
          class="controls__file__save"
          @click="saveToLibrary"
        >
          Сохранить файл в библиотеку
        </v-btn>
        <v-btn
          v-else
          class="controls__file__save"
          disabled
        >
          Файл в библиотеке
        </v-btn>

        <GenerateModal
          @generate="generate"
        />
        <ActionsModal
          v-model="showActionsModal"
          :click-x="(clickX as number)"
          @add-event="addEvent"
          @crop="crop"
        />
      </div>
      <div
        v-if="sor"
        class="controls__settings"
      >
        <div class="controls__settings__naive">
          <v-text-field
            v-model="reflThr"
            label="Лимит отражения (дБ)"
            @update:model-value="updateRefl"
          />
          <v-text-field
            v-model="lossThr"
            label="Лимит неотражения (дБ)"
            @update:model-value="updateLoss"
          />
          <v-text-field
            v-model="teoThr"
            label="Лимит Тигера (дБ)"
          />
        </div>
        <div class="controls__settings__teo">
          <v-checkbox
            v-model="showBuiltin"
            :disabled="sorEdited"
          >
            <template #label>
              Встроенные события
              <div
                class="colorIndicator"
                style="background: black;"
              ></div>
            </template>
          </v-checkbox>
          <v-checkbox
            v-model="showNaive"
          >
            <template #label>
              События найденные простым методом
              <div
                class="colorIndicator"
                style="background: blue;"
              ></div>
              <div
                class="colorIndicator"
                style="background: red;"
              ></div>
            </template>
          </v-checkbox>
          <v-checkbox
            v-model="showTeo"
          >
            <template #label>
              События найденные с помощью TEO
              <div
                class="colorIndicator"
                style="background: green;"
              ></div>
            </template>
          </v-checkbox>
          <v-checkbox
            v-model="showTeoVal"
          >
            <template #label>
              График TEO
              <div
                class="colorIndicator"
                style="background: pink;"
              ></div>
            </template>
          </v-checkbox>
        </div>
      </div>
    </div>
    <v-btn
      v-else
      class="openFile"
      color="primary"
      @click="openFile"
    >
      Открыть файл
    </v-btn>
  </v-main>
</template>

<style scoped lang="scss">
.openFile {
  top: 50%;
  left: 50%;
  transform: translate(-50%, -50%);
}

.controls {
  padding: 30px;
  display: flex;
  gap: 10px;
}

.controls__file,
.controls__settings {
  display: flex;
}

.controls__file {
  gap: 10px;
  width: 300px;
  flex-direction: column;
}

.controls__file__filename {
  display: block;
  overflow: hidden;
  text-overflow: ellipsis;
}

.controls__settings {
  flex-direction: row;
  flex-grow: 1;
  flex-wrap: wrap;
  gap: 10px;

  > * {
    flex-grow: 1;
    box-sizing: border-box;
    width: calc(50% - 10px);
  }
}

.colorIndicator {
  width: 16px;
  min-width: 16px;
  height: 16px;
  border: 1px solid black;
  margin-left: 8px;
}
</style>  
\end{lstlisting}
