\makenoidxglossaries

\newglossaryentry{реактивность}{
  name=реактивность,
  description={способность фрагмента кода автоматически обновляться или перерисовываться (реагировать) в ответ на изменения в данных, к которым он привязан}
}
\newglossaryentry{транспиляция}{
  name=транспиляция,
  description={преобразование программы, написанной на одном языке программирования в качестве исходных данных, в эквивалентный код другой версии этого языка или в другой язык программирования того же уровня абстракции}
}
\newglossaryentry{вид}{
  name=вид,
  description={состояние приложения которое определяется набором и расположением представленных элементов графического интерфейса}
}
\newglossaryentry{модальное окно}{
  name={модальное окно},
  description={окно, которое блокирует работу пользователя, но оставляет главный экран видимым вместе с модальным окном. Пользователи должны взаимодействовать с модальным окном, прежде чем смогут вернуться в родительское приложение}
}
\newglossaryentry{фреймворк}{
  name=фреймворк,
  description={программное обеспечение, которое описывает правила построения архитектуры приложения, задавая на начальном этапе разработки поведение по умолчанию --- <<каркас>>, который нужно расширять и изменять, согласно некоторым требованиям}
}

\newacronym{волс}{ВОЛС}{волоконно-оптическая линия связи}
\newacronym{ор}{ОР}{оптический рефлектометр во временной области}
\newacronym{мор}{МОР}{метод обратного рассеяния}
\newacronym{сор}{СОР}{сигнал обратного рассеяния}
\newacronym{фпу}{ФПУ}{фотоприемное устройство}
\newacronym{лвт}{ЛВТ}{линейный волоконный тракт}
\newacronym{ов}{ОВ}{оптическое волокно}
\newacronym{teo}{TEO}{энергетический оператор Тигера (Teager energy operator)}
\newacronym{otdr}{OTDR}{Optical time domain reflectometer (оптический рефлектометр)}
\newacronym{по}{ПО}{программное обеспечение}
